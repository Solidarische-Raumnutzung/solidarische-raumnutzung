%Dies ist die Hauptseite des Dokumentes. Es werden u. a. alle Kapitel,
%Einstellung im Header eingebunden.  Veränderungen müssen in folgenden Dateien
%vorgenommen werden:
      %- config.tex
      %- einzelne Kapitel (evtl. erweitern)

\input{common/layout}           % Diese Datei enthält alle
                                           % Layouteinstellungen
\newcommand{\dokumentTitel}{Pflichtenheft}
% Definition von globalen Parametern, die derzeit auf der Titelseite und in der
% Kopfzeile verwendet werden. Der in <> gesetzte Text ist zu verändern.

\newcommand{\praktikumTitel}{Praxis der Softwareentwicklung}
\newcommand{\projektTitel}{Solidarische Raumnutzung}

\newcommand{\semester}{Wintersemester 2024/25}
\newcommand{\institut}{
	Institut für Anthropomatik und Robotik (IAR)\\
	Forschungsgruppe Mensch-Maschine-Interaktion und Barrierefreiheit (MBI)\\
	Prof. Dr. Kathrin Gerling\\
	Adenauerring 10, Gebäude 50.28\\
	76131 Karlsruhe\\
}
\newcommand{\institutsLogo}{common/hci.jpg}
\newcommand{\kitlogo}{common/kit_logo.jpg}
\newcommand{\betreuer}{Sabrina Burtscher, Dmitry Alexandrovsky}


\newcommandx{\unsure}[2][1=]{\todo[linecolor=red,backgroundcolor=red!25,bordercolor=red,#1]{#2}}
\newcommandx{\change}[2][1=]{\todo[linecolor=blue,backgroundcolor=blue!25,bordercolor=blue,#1]{#2}}
\newcommandx{\info}[2][1=]{\todo[linecolor=OliveGreen,backgroundcolor=OliveGreen!25,bordercolor=OliveGreen,#1]{#2}}
\newcommandx{\improvement}[2][1=]{\todo[linecolor=Plum,backgroundcolor=Plum!25,bordercolor=Plum,#1]{#2}}
\newcommandx{\thiswillnotshow}[2][1=]{\todo[disable,#1]{#2}}

\makeglossaries

%------Beginn des Gesamtdokumentes----------------------------------------------
\begin{document}

%------Eingebundene Seiten, Verzeichnisse bzw. Kapitel--------------------------
\include{common/titelseite}                       % Titelseite

%!TEX root = ../Pflichtenheft.tex

\chapter{Glossar}

Hier werden Fachbegriffe erklärt.


%----Bearbeiterübersicht--------------------
%Diese Tabelle ist eine Übersicht, wer welche Teile des Dokumentes verfasst hat.
%Sie ist vollständig zu bearbeiten.
{\relsize{2}\textbf{Bearbeiterübersicht}}\\[2ex]
\begin{longtable}{|m{3.38cm}|m{4.0cm}|m{6.0cm}|}
  \hline                                              % Gitterlinie oberhalb
  \textbf{Kapitel}  &    \textbf{Autoren}  &    \textbf{Kommentare} \\ 
  \hline \hline                                       % Gitterlinie
  
  %----Die Granularität sollte mindestens auf section level sein.
  %----Fehlende sections bitte hier nachtragen. Sollte eine feinere Granularität
  %----nötig sein, sollte dies hier ebenfalls nachgetragen werden.
  1   &    ...    &    ...  \\ \hline
  1.1 &    ...    &    ...  \\ \hline
  1.2 &    ...    &    ...  \\ \hline
  2   &    ...    &    ...  \\ \hline
  3   &    ...    &    ...  \\ \hline
  3.1 &    ...    &    ...  \\ \hline
  3.2 &    ...    &    ...  \\ \hline
  4   &    ...    &    ...  \\ \hline
  4.1 &    ...    &    ...  \\ \hline
  4.2 &    ...    &    ...  \\ \hline
  4.3 &    ...    &    ...  \\ \hline
  5   &    ...    &    ...  \\ \hline
  5.1 &    ...    &    ...  \\ \hline
  5.2 &    ...    &    ...  \\ \hline
  5.3 &    ...    &    ...  \\ \hline
  6   &    ...    &    ...  \\ \hline
  6.1 &    ...    &    ...  \\ \hline
  6.2 &    ...    &    ...  \\ \hline
  6.3 &    ...    &    ...  \\ \hline
  7   &    ...    &    ...  \\ \hline
\end{longtable}

\tableofcontents                           % Inhaltsverzeichnis wird automatisch
                                           % generiert
\listoffigures              
\newpage
\listoftodos[Notes]
%----Kapitel des Pflichtenhefts, die mit Inhalt zu füllen sind--------------------
%!TEX root = ../Pflichtenheft.tex
\chapter{Einleitung}
\todo[inline]{Hier müssen definitiv ein paar einleitende Worte hin...}
Am Institut... idk...
%!TEX root = ../Pflichtenheft.tex

\chapter{Zielbestimmung}
\label{chap:target}


\section{Musskriterien}\label{sec:musskriterien}

\must{1}{Die Anwendung muss als Web-Applikation realisiert werden.}
\must{2}{Nutzende der Anwendung müssen sich mit ihrem KIT-Konto per \gls{OIDC} oder einem lokalen Gastkonto anmelden können.}
\must{3}{Nutzende müssen sich abmelden können.}
\must{4}{Die Anwendung muss die Ansichten Kalender, Termin, Termin-erstellen, Login, Kontenliste und Terminübersicht anbieten.}
\must{5}{Die Ansicht \textit{Kalender} muss einen klaren Überblick über die bereits reservierten Zeiten geben.}
\must{6}{Die Ansicht \textit{Kalender} muss die Öffnungszeiten des Raumes darstellen.}
\must{7}{Die Ansicht \textit{Kalender} muss die Termine des/r angemeldeten Nutzenden hervorgehoben darstellen.}
\must{8}{Die Ansicht \textit{Termin} muss die Möglichkeit bieten, genauere Informationen über einen Termin darzustellen.}
\must{9}{Die Ansicht \textit{Termin-Erstellen} muss die Möglichkeit bieten, einen Raum für eine bestimmte Zeitperiode zu reservieren.}
\must{10}{Bei der Reservierung eines Raumes muss optional die Möglichkeit bestehen eine Beschreibung zu hinterlegen. Hierbei müssen die Nutzenden klar darauf hingewiesen werden, wer diese Daten einsehen kann.}
\must{11}{Die Terminübersicht muss den Nutzenden die Möglichkeit bieten, all ihre Termine zu verwalten.}
\must{12}{Die Priorität eines Termins ist in drei Stufen gegliedert.}
\must{13}{Beim Erstellen eines Termins wählen die Nutzenden aus, ob sie ihren Raum mit anderen Personen teilen möchten. Dabei stehen die Optionen \textit{Ja}, \textit{Nein} und \textit{Auf Anfrage} zur Auswahl.}
\must{14}{Die Anwendung muss Terminkonflikte reibungslos mithilfe der Prioritäten lösen können. Dabei überschreiben Termine mit höherer Priorität andere.}
\must{15}{Die Anwendung muss in der Lage sein, Nutzende per E-Mail darüber zu informieren, wenn ihr Termin durch einen Termin mit höherer Priorität überschrieben wurde.}
\must{16}{Nutzende der Anwendung müssen in der Lage sein, eine Reservierung zu stornieren.}
\must{17}{Es muss ein Adminkonto geben, welches per Passwort authentifiziert wird. Nur der Server-Admin darf dieses Passwort ändern können.}
\must{18}{Die Ansicht \textit{Kontoliste} muss den Admins die Möglichkeit bieten, einzelne Konten sowie die Anmeldung per Gastkonto zu deaktivieren.}
\must{19}{Admins müssen in der Ansicht \textit{Termin} Termine löschen können.}
\must{20}{Es muss ein farbcodiertes Banner geben, der den aktuellen Status des Raumes anzeigt.}
\must{21}{Admins müssen in der Ansicht \textit{Kalender} die Öffnungszeiten einstellen können}


\section{Wunschkriterien}\label{sec:wunschkriterien}

\wish{1}{Es könnte die Möglichkeit geben, mehr als einen Raum zur Buchung anzubieten. Dabei könnte eine Raumauswahl vor der Ansicht \textit{Kalender} die verschiedenen Möglichkeiten präsentieren. Existiert nur ein Raum, wird diese Auswahl übersprungen.}
\wish{2}{In der Ansicht \textit{Kalender} könnten Feiertage automatisch eingebunden werden.}
\wish{3}{Admins könnten die Möglichkeit haben, geplante Wartungs- und Sperrzeiten einzurichten.}
\wish{4}{Termine könnten nach der Buchung im \gls{iCal}-Format zum Export angeboten werden.}
\wish{5}{Nutzende könnten in der Ansicht \textit{Termin} die Möglichkeit haben, ihre eigenen Termine zu bearbeiten.}
\wish{6}{Die Ansicht \textit{Kalender} könnte visualisieren, welche Termine bereits in der Vergangenheit liegen und wo der Übergang von der Vergangenheit zur Zukunft liegt.}
\wish{7}{Tooltips könnten Nutzenden erklären, wofür bestimmte Elemente der \gls{UI} verwendet werden.}
\wish{9}{Es könnte einen physischen Panik-Button geben.}
\wish{10}{Die Anwendung könnte in der Lage sein, Nutzende zu informieren, falls ein gewünschter Termin frei wird.}
\wish{11}{Es könnte einen Quick-Checkin-Button geben, welcher eine vorausgefüllte Terminerstellung öffnet. Dieser bietet auch eine Alternative zur Interaktion mit dem Kalender.}
\wish{12}{Es könnte einen Quick-Checkout-Button geben, der vorzeitiges Beenden eines Termines ermöglicht.}
\wish{13}{Admins könnten eine Statistik-Ansicht nutzen.}


\section{Abgrenzungskriterien}\label{sec:abgrenzungskriterien}

\wont{1}{Die Verteilung von Buchungen zwischen ähnlichen Räumen ist nicht Teil des Projekts.}
\wont{2}{Das Skalieren der Anwendung auf eine große Anzahl von Räumen ist nicht vorgesehen.}
\wont{3}{Die Buchung von Räumen für mehrere Tage ist nicht vorgesehen.}
\wont{4}{Die Verwaltung von Räumen, die mehrere Arbeitsplätze umfassen, ist nicht vorgesehen.}
\wont{5}{Die Entwicklung plattformspezifischer Anwendungen und der dafür notwendigen \gls{API}s ist nicht vorgesehen.}
\wont{6}{Die Reservierung ist nur für die nahe Zukunft gedacht, eine Langzeitplanung ist nicht vorgesehen.}
\wont{7}{Die Reservierung wiederholter Termine ist nicht vorgesehen.}
\wont{8}{E-Mail-Adressen von Gastkonten werden nicht verifiziert.}
%!TEX root = ../Pflichtenheft.tex

\chapter{Produktübersicht}
\label{chap:product_overview}

\section{Betriebsbedingungen}
\begin{itemize}
    \item Die Software soll in einem \gls{Docker}-\gls{Container} laufen.
    \item Die Software soll weitgehend ohne Unterbrechung laufen.
    \item Ständige Beobachtung ist nicht vorgesehen, die Software soll weitgehend autonom laufen.
    \item Übliche Administrationsaufgaben sollen über die Administrationsoberfläche und ohne Neustart der Software möglich sein.
    \item Die \gls{RAM}-Nutzung soll möglichst gering sein (Ideal: unter 1GB)
\end{itemize}

\section{Use-Cases}

Der folgende Abschnitt hat die Aufgabe, die Funktionalität des zu entwickelnden
Systems grafisch mithilfe von Use-Case-Diagrammen und einer kurzen verbalen
Beschreibung zu charakterisieren. Im Erklärungstext sollte darauf eingegangen werden,
welcher Use-Case welches Kriterium erfüllt. Es sind so viele Use-Case-Diagramme einzufügen,
wie zur vollständigen und übersichtlichen Beschreibung der Systemfunktionalität
notwendig sind.

Anmerkungen zu Use-Cases:

\begin{itemize}
	\item Der Zweck eines Use-Cases ist es, zu beschreiben wie ein Aktor das System benutzen kann,
	 	um ein bestimmtes Ziel zu erreichen.
	\item Ein Use-Case ist nicht immer eine Funktion, sondern das Ziel des Systems und kann mehrere 
		Funktionen umfassen.
		In diesem Fall, kann man dann die einzelnen Funktionen in Aktivitätsdiagrammen beschreiben.
	\item Use-Cases werden klassischerweise so benannt, wie die Ziele aus Sicht der Akteure heißen.
	\item Das Diagramm soll nur den groben Aufbau des Systems beschreiben, damit man sehen kann,
		 ob das System das Richtige tut oder nicht.
\end{itemize}


Beispiel:

\begin{figure}[ht]
\centering
\caption{Use-Case-Diagramm Buchungssystem}
\end{figure}

Dieses Diagramm zeigt ein minimales Use-Case-Diagramm für ein Buchungssystem.
Der Akteur, hier ein Benutzer der einen Flug buchen möchte, kann einen Flug buchen oder
sein Benutzerkonto ansehen. Diese beiden Use-Cases sind für den Benutzer zielführend.
Ein Use-Case \emph{Flug suchen} wäre hier falsch, denn das ist nicht das Ziel eines Buchungssystems. 
Der Use-Case \emph{Flug buchen} beinhaltet mehrere Funktionen des zugehörigen Systems. 
Diese wären \emph{Einloggen}, \emph{Flug reservieren} usw. und sollen im nächsten Kapitel beschrieben werden. 
In diesem Fall wäre nun ein Aktivitätsdiagramm zur Beschreibung der einzelnen Funktionen notwendig.


Interessante Use-Cases können mit einem Aktivitätsdiagramm genauer erkläutert werden.

%!TEX root = ../Pflichtenheft.tex

% Kapitel 4
%-------------------------------------------------------------------------------

\chapter{Produktfunktionen}
\label{chap:product_functions}

In Abhängigkeit von den gewählten Konzepten erfolgt hier eine Konkretisierung
und Detaillierung der Funktionen aus den Use-Case-Diagrammen und ggf. dem
Angebot.

Die Produktfunktionen müssen die Kriterien aus den Zielbestimmungen abdecken.
Dabei kann es je nach Kriterium eine oder mehrere Funktion geben.
Das nachfolgende Format sollte für einige Funktionsbeschreibungen übernommen werden:

Beispiel:\\
\begin{function}{10}{Lagerverwaltung}
\item[Anwendungsfall:] Automatisches Einlagern
\item[Anforderung:] \lfk{20} (Wenn kein Lastenheft vorhanden, dann Kriterium aus den Zielbestimmungen, z.B.: \ref{RM1})
\item[Ziel:] Ein Reifen erscheint am Systemeingang (Scanner), erhält einen Lagerplatz
zugewiesen und wird dort eingelagert.
\item[Vorbedingung:] Das Scannen des Barcode-Reifens muss erfolgreich sein, sonst kann
der Typ nicht ermittelt werden. Solche unbekannten Reifen werden direkt in den
Überlauf gefördert.
\item[Nachbedingung Erfolg:] Reifen ist physikalisch eingelagert und logisch in der
Datenbank verbucht.
\item[Nachbedingung Fehlschlag:] Der Reifen wurde infolge gestörter Fördermechanik
nicht eingelagert (liegt im Überlauf) oder produzierte aufgrund inkonsistenter
Datenbank einen "`Platz belegt"' - Fehler beim Anfahren eines irrtümlich
als frei angenommenen Platzes.
\item[Akteure:] ~Produktion
\item[Auslösendes Ereignis:] SPS meldet der Steuerung, dass am Eingangsscanner ein
Reifen mit Seriennummer X des Typs Y eingetroffen ist.
\item[Beschreibung:] ~
\begin{enumerate}
  \item Reifentypinformationen ermitteln (besonders Höhe des Reifens bei Wahl zwischen unterschiedlich hohen Lagerplätzen wichtig).
  \item Alle Module ermitteln, die\\
-  Platz auf den Einlagerstichen haben\\
-  momentan nicht im Störungszustand sind\\
-  freie Lagerplätze in der geforderten Höhe aufweisen.
  \item Lagerplatz nach Gleichverteilungsgrundsatz bestimmen.
  \item Reifen auf den Einlagerstich des gewählten Moduls befördern.
  \item Sobald er auf dem vordersten Platz des Einlagerstichs steht, dem Modul den Befehl zur Reifenaufnahme und Einlagerung auf den gewählten Platz schicken.
\end{enumerate}
\item[Erweiterung:] (optional)\\
	2a Zur Effizienzsteigerung auch Module ansteuern, die momentan keinen Platz auf
	den Einlagerstichen haben, aber wahrscheinlich so schnell einlagern, dass der
	Reifen nach der Fahrtzeit zum Modul auf den Stich eingelagert werden kann
	(Überwachung des "`Unterwegsbestandes"' an Reifen für ein bestimmtes
	Modul).\\
	3a Lagerplatz des Reifens möglichst nah zum Einlagerstich im RBG wählen	(kürzere RBG-Fahrtzeiten).
\item[Alternativen:] (optional)\\
	2a Wenn kein Lagerplatz gefunden wird, Reifen zum Überlauf schicken (der
	Einlagerförderer wird niemals angehalten!).
\end{function}

\begin{function}{20}{Login}
\item[Anwendungsfall:] ...
\end{function}
%!TEX root = ../Pflichtenheft.tex

\chapter{Produktdaten}
\label{chap:product_data}

\iffalse
Die langfristig zu speichernden Daten sind aus Benutzersicht detaillierter zu beschreiben.
Dabei bietet sich eine formale Beschreibung an, um eine größere Präzisierung zu erreichen.
Es sollte eine Menge an erwarteten Daten angegeben werden.

Es kann die Darstellung gemäß Beispiel verwendet werden (alternativ kann auch ein Klassendiagramm mit entsprechender Beschreibung erstellt werden):\\

\begin{data}{10}{Lagerdaten}
	Daten der Lagerplätze (max. 5.000):\\
	-  Modulnummer,\\
	-  Regalseite,\\
	-  Regalspalte,\\
	-  Regalzeile,\\
	-  Fachhöhe,\\
	-  Platzsperre (0 = nicht gesperrt, 1 = gesperrt für Einlagerung, 2 = gesperrt
	   für Auslagerung, 3 = gesperrt für alle Zugriffe),\\
	-  Reifenstatus (0 = frei,1 = reserviert für Einlagerung, 2= belegt, 3 =
	   reserviert für Auslagerung),\\
	-  Reifenseriennummer.\\
\end{data}

\begin{data}{20}{Moduldaten}
	Daten der Module (max. 20):\\
	-  Modulnummer,\\
	-  Sperrkennzeichen (0 = nicht gesperrt, 1 = gesperrt für Einlagerung, 2 =
	   gesperrt für Auslagerung, 3 = gesperrt für alle Zugriffe),\\
	-  maximale Kapazität,\\
	-  freie Kapazität,\\
	-  belegte Plätze (ergibt sich aus Status und Zahl der zugeordneten
	   Lagerplätze, wird aus Geschwindigkeitsgründen allerdings redundant
	   mitgeführt).
\end{data}
\fi

Die Anwendung verwendet den Server als zentralen Speicherort für alle Daten.
Die Daten werden in einer \gls{PostgreSQL}-Datenbank gespeichert.
Auf dem Client werden nur temporäre Daten gespeichert, die für die Funktionalität der Anwendung notwendig sind.

\improvement{Das LaTeX-Template bietet für diese Sektion ein anderes Format an. Ich hab es nicht schön gekriegt, deshalb habe ich es so gelöst.}

\subsection*{Clientdaten}
\begin{itemize}
    \item Anmeldungscookie (falls der Benutzer anonym angemeldet ist)
    \item Zustand der Anwendung (z.B.\ aktuelle Seite, geöffnete Dialoge)
\end{itemize}

\subsection*{Serverdaten}
\begin{itemize}
    \item Benutzerdaten
    \begin{itemize}
        \item Benutzername (oder \textit{Anonym} für anonyme Benutzer)
        \item (optional) E-Mail-Adresse
        \item OAuth- oder Cookie-Token
    \end{itemize}
    \item Ereignisdaten
    \begin{itemize}
        \item Start- und Endzeitpunkt \unsure{Wie kodieren wir die Zeit? Fixe Slots oder Unix-Timestamp?}
        \item Beschreibung
        \item Raum
        \item Ersteller
        \item Priorität
        \item Kollaborativität
        \item (optional) E-Mail-Adresse
        \item Sichtbarkeit der E-Mail-Adresse
    \end{itemize}
    \item Raumdaten
    \begin{itemize}
        \item Raumname
        \item Raumbeschreibung
        \item Raumbild
    \end{itemize}
\end{itemize}
%!TEX root = ../Pflichtenheft.tex

% Kapitel 6
%-------------------------------------------------------------------------------

\chapter{Nichtfunktionale Anforderungen}
\label{chap:non_functional_req}

In diesem Kapitel wird festgelegt, welche Qualitätsmerkmale das zu entwickelnde
Produkt in welcher Qualitätsstufe besitzen soll. Anschließend werden die als am
wichtigsten bezeichneten Qualitätsmerkmale operationalisiert, d.h. in konkrete
Produktanforderungen detailliert, falls sie nicht als allgemeine Richtlinie (z.
B. Standard, Norm) zur Verfügung gestellt werden können.


Die oben als am wichtigsten bezeichneten Qualitätsmerkmale werden im Folgenden
operationalisiert, d.h. in konkrete Produktanforderungen detailliert oder es
wird angegeben, welche Richtlinie (z. B. Standard, Norm) einzuhalten ist. Diese
Qualitätsanforderungen werden wie im Beispiel definiert. Zu prüfen ist, ob die
gewünschte Qualität mit den in Produktdaten genannten Datenmengen erreicht
werden kann.


Beispiele:

\notfunctional{1}{Die Anwendung soll schnell und einfach bedienbar sein.}
\notfunctional{2}{Die Anwendung soll auf mobilen sowie Desktop-Endgeräten angepasst laufen.}
\unsure{Kann man hier noch weiter spezifizieren?}
\notfunctional{2}{Die Anwendung soll in allen gängigen Browsern lauffähig sein. Insbesondere beinhaltet dies Firefox und Chrome.}
\notfunctional{5}{Es muss die Anonymität der Nutzer gewährleistet sein.}
\notfunctional{4}{Es soll entsprechend auf Acessibility geachtet werden.}

%!TEX root = ../Pflichtenheft.tex

\chapter{Benutzeroberfläche}
\label{chap:ui}

In diesem Kapitel kann, an euer Projekt angepasst, die Benutzeroberfläche durch UI Mocks beschrieben werden.



%!TEX root = ../Pflichtenheft.tex

% Kapitel 10
% Die Unterkapitel können auch in separaten Dateien stehen,
% die dann mit dem \include-Befehl eingebunden werden.
%-------------------------------------------------------------------------------

\chapter{Technische Produktumgebung}
\label{chap:tech_env}

In diesem Kapitel wird die technische Umgebung des Produktes beschrieben. Bei
Client/Server-Anwendungen ist die Umgebung jeweils für Client und Server
getrennt anzugeben.

\section{Software}
Hier wird angegeben, welche Softwaresysteme (z. B. Betriebssystem, Datenbank,
Fenstersystem, usw.) zur Verfügung stehen.\\

Beispiel:\\
Server-Betriebssystem: Linux (Debian 7)\\
Client-Betriebssystem: Windows 7\\


\section{Hardware}
Hier werden die Hardware Komponenten (z. B. CPU, Peripherie) in minimaler und
maximaler Konfiguration aufgeführt, die für den Produkteinsatz vorgesehen
sind.

Beispiel:\\
Server: Virtueller Server mit 4 Cores und 4 GB RAM \\
Client: Standard PC und browserfähiges Gerät mit Grafikbildschirm (für Fernwartung)\\

\chapter{Testfälle und Testeszzenarien}
\label{chap:test}
In diesem Kapitel definieren wir die Testfälle und Testfallszenarien.

\section{Basis-Testfälle}

Jeder Produktfunktion entspricht einem Basis-Testfall. Die Basis-Testfälle sind uten aufgelistet.


\begin{table}[htbp]

  \centering
%    \caption{Überblick von allen Funktionen.}
  \begin{tabularx}{\textwidth}{ l|X|l }
      \textbf{Nr.} & \textbf{Beschreibung} & \textbf{Funktion} \\ \hline\hline
      ⟨T10⟩ & Landeseite besuchen &\ref{F10}\\
      ⟨T20⟩ & Login &\ref{F20} \\
      ⟨T30⟩ & Abmelden &\ref{F30} \\
      ⟨T40⟩ & Reservieren &\ref{F40} \\
      ⟨T50⟩ & Löschen von Terminen durch Administratoren &\ref{F50} \\
      ⟨T60⟩ & Deaktivieren eines Kontos &\ref{F60} \\
      ⟨T70⟩ & Benachrichtigung bei freiem Raum &\ref{F70} \\
      %⟨T80⟩ & Anzeige des Raumstatus &\ref{F80} \\'
      ⟨T90⟩ & Stornierung einer Reservierung &\ref{F90} \\
      ⟨T100⟩& Login mit Adminkonto &\ref{F100} \\
      ⟨T110⟩ & Deaktivierung von Gastkonten &\ref{F110} \\
      ⟨T120⟩& Öffnungszeiten einstellen &\ref{F120} \\
      ⟨T130⟩& Terminkonfliktauflösung &\ref{F130} \\
  \end{tabularx}\label{tab:test_table}
\end{table}

\pagebreak

\section{Testfallszenarien}
Die Testfallszenarien ergeben sich als Komposition der Basis-Testfälle.\\ \\
\begin{scenario}{10}{Besuch der Landeseite und Anmeldung/Abmeldung}
  \item[Ziel:] Sicherstellen, dass Nutzende die Landeseite aufrufen und sich erfolgreich anmelden bzw.\ abmelden können.
  \begin{enumerate}
    \item Der Nutzende besucht die Landeseite ⟨T10⟩.
    \item Der Nutzende loggt sich ein ⟨T20⟩.
    \item Der Nutzende meldet sich ab ⟨T30⟩.
  \end{enumerate}
\end{scenario}

\begin{scenario}{20}{Reservierung eines Raums und Stornierung}
  \item[Ziel:] Überprüfen, ob Nutzende erfolgreich Termine reservieren können.
  \begin{enumerate}
    \item Der Nutzende besucht die Landeseite ⟨T10⟩.
    \item Der Nutzende meldet sich an ⟨T20⟩.
    \item Der Nutzende wählt einen verfügbaren Raum aus und reserviert diesen ⟨T40⟩.
    \item Buchen ausserhalb der Öffnungszeiten sollte hierbei fehlschlagen.
    \item Der Nutzende storniert die Reservierung ⟨T90⟩.
    \item Andere Nutzende, welchen sich diesen Termin vorgemerkt hatten, werden benachrichtigt, dass der Raum nun Frei ist ⟨T70⟩.
  \end{enumerate}
\end{scenario}

\begin{scenario}{30}{Verwaltung durch das Adminkonto}
  \item[Ziel:] Testen der administrativen Funktionalitäten zum Löschen von Terminen, Deaktivieren von Gastkonten und einstellen der Öffnungszeiten.
  \begin{enumerate}
    \item Ein Admin besucht die Landeseite ⟨T10⟩.
    \item Der Admin meldet sich an ⟨T100⟩.
    \item Der Admin löscht bestehende Termine ⟨T50⟩.
    \item Der Admin deaktiviert die Anmeldung von Gastkonten ⟨T110⟩.
    \item Der Admin ändert die Öffnungszeiten für einen bestimmten Wochentag ⟨T120⟩.
    \item Der Admin öffnet die Ansicht \textit{Kontoliste} und deaktiviert ein Konto ⟨T60⟩.
    \item Der Admin meldet sich ab ⟨T30⟩.
    \item Das Anmelden mit einem Gastkonto sollte jetzt fehlschlagen ⟨T20⟩.
  \end{enumerate}
\end{scenario}

\todo{this is just wrong!}
\begin{scenario}{40}{Termin anzeigen}
  \item[Ziel:] Überprüfen, ob ein Termin richtig.
  \begin{enumerate}
    \item Der Nutzende besucht die Landeseite ⟨T10⟩.
    \item Der Nutzende meldet sich an ⟨T20⟩.
    \item Der Nutzende wählt einen Raum aus und überprüft dessen Status ⟨T20⟩.
  \end{enumerate}
\end{scenario}

\begin{scenario}{50}{Terminkonflikt auflösung}
  \item[Ziel:] Überprüfen, ob ein Terminkonflikt richtig aufgelöst wird.
  \begin{enumerate}
    \item Konto 1 meldet sich an ⟨T20⟩.
    \item Konto 1 erstellt einen Termin, und gibt die zu testende Priorität, Raumteilungsoption und Zeitperiode ein ⟨T20⟩.
    \item Konto 1 meldet sich ab und Konto 2 meldet sich an.
    \item Konto 2 erstellt einen Termin, welcher den anderen überlappt.
    \item Es wird überprüft, ob die erwartete Konfliktauflösung stattgefunden hat und ggf.\ die dafür benötigten E-Mails versendet wurden ⟨T130⟩.
  \end{enumerate}
\end{scenario}








\setglossarystyle{german}
\printglossaries

%------Ende des Dokumentes------------------------------------------------------
\end{document}
