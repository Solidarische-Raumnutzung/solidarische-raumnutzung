%!TEX root = ../Pflichtenheft.tex

\chapter{Glossar}

\begin{description}
\item[CI/CD] Baut automatisiert Software, führt Tests durch und veröffentlicht Artefakte
\item[Docker] Software zur Bereitstellung von Anwendungen innerhalb von Containern
\item[Container] Isolierte Umgebung um Software unabhängig von der zugrunde liegenden Umgebung auszuführen
\item[Browser] Software zum Navigieren von Webseiten, zum Beispiel Firefox oder Chrome
\item[Git] Software zur Versionsverwaltung von Softwareprojekten
\item[GitHub] Plattform zur Versionsverwaltung von Softwareprojekten, nutzt Git
\item[IDE] (Integrated Development Environment) Software, welche alle Werkzeuge zur Softwareentwicklung in einem Programm kombiniert
\item[AMD64] Verbreitete Prozessorarchitektur von Intel und AMD
\item[RAM] (Random Access Memory) Arbeitsspeicher
\item[VM] (Virtuelle Maschine) Software zur Simulation eines Computers
\item[PostgreSQL] Objekt-Relationales Datenbankmanagementsystem, welches zum Speichern und Verwalten von Daten verwendet wird
\item[HTML] (Hypertext Markup Language) Eine Sprache um die Struktur und den Inhalt einer Website zu definieren
\item[CSS] Cascading Style Sheets, ist eine Sprache um das Visuelle aussehen einer Website zu definieren
\item[JavaScript] Programmiersprache um das Logische verhalten von Webseiten zu steuern
\item[SSR] (Server-Side Rendering) Methode um das HTML einer Website auf dem Server zu produzieren, anstelle des Browsers
\item[REST] (Representational State Transfer) Architekturstil von APIs für das Internet
\item[API] Schnittstelle auf Quelltext-Ebene um anderen Programmen funktionen zur Verfügung zu stellen
\item[OIDC] (OpenID Connect) Authorisierungsframework welches vom KIT genutzt wird um dritten Webseiten Logins basierend auf KIT-Konten bereitzustellen
\end{description}