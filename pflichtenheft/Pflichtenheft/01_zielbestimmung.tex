%!TEX root = ../Pflichtenheft.tex
\section{Allgemeine Hinweise}
\textbf{Template} Tragen Sie vor Beginn der Ausarbeitung Ihre Daten in die Datei common/teilnehmer.tex ein. Löschen Sie die Template-Texte und -Bilder, sobald Sie Ihren Text geschrieben haben.

\textbf{Anglizismen} Verwenden Sie ruhig die englischen Wörter.

\textbf{Fett-/Kursivschreibweise} Entscheiden Sie sich zu Beginn innerhalb der Gruppe, wann Sie \textit{kursive} und wann Sie \textbf{fette} Schrift einsetzen möchten.

\textbf{Bilder} Legen Sie eigene Bilder stets im figures Ordner ab. Nutzen Sie die im Template beispielhaft eingebundenen Bilder als Vorlage. Außerdem erlaubt Overleaf das Einfügen von Bildern im GUI (siehe obere Task-Leiste).


\chapter{Zielbestimmung}
\label{chap:target}
\improvement[inline]{Dieser Abschnitt sollte noch ausführlicher beschrieben werden.}
\iffalse
Dieser Abschnitt hat die Aufgabe, als eine Art Einleitung zu dienen. Es soll
ein kurzer Umriss über Ziel und Motivation des Gesamt- und ggf. der
Teilprojekte dargestellt werden. Beschrieben wird die Hauptaufgabe des Systems.
Wichtig ist, den Grund für die Systementwicklung (Probleme oder Geschäftsidee)
und damit ihre Ziele herauszuarbeiten.
\fi

\section{Musskriterien}\label{sec:musskriterien}
In diesem Abschnitt halten wir unabdingbare Leistungen der Software fest\\
Wir halten also welche Funktionalitäten/Leistungen das Softwareprodukt in
jedem Fall erfüllen muss, damit es genutzt werden kann.

\todo[inline]{Hier fallen uns noch sicherliche mehr Kriterien ein.}
\must{1}{Die Anwendung muss als Web-Applikation realisiert werden.}
\must{2}{Nutzer der Anwendung müssen in der Lage sein, sich mit ihrem KIT-Konto per \gls{OIDC} oder einem lokalen Gastkonto anzumelden.}
\must{3}{Es müssen die Ansichten Kalender, Termine und Termin-erstellen die Benutzerinteraktionen ermöglichen.}
\must{4}{Die Kalender-Ansicht muss einen klaren Überblick über bereits reservierte Zeiten geben.}
\must{5}{Die Kalender-Ansicht muss die Öffnungszeiten des Raumens zur Darstellung bringen.}
\must{6}{Die Kalender-Ansicht soll angemeldeten Nutzern ihre eigenen Termine hervorgehoben kennzeichnen.}
\must{7}{Die Termin-Ansicht muss die Möglichkeit bieten, genauere Informationen über einen Termin darzustellen.}
\must{8}{Die Termin-Erstellen-Ansicht muss die Möglichkeit bieten, einen Raum für eine gewisse Zeitperiode zu reservieren.}
\unsure{ist Kollaborativ ein guter Name dafür??}
\must{9}{Ereignisse müssen als kollaborativ gekennzeichnet werden können. Dies muss auf der Kalenderansicht ersichtlich sein.}

\unsure{what exactly should be specified??}
\must{10}{Bei der Reservierung eines Raumes muss die Möglichkeit bestehen, eine Beschreibung, Namen, Mail-Adresse TODO... zu hinterlegen. Hierbei muss der Nutzer klar darauf hingewiesen werden, wer diese Daten einsehen kann.}

\unsure{Specify more concretely, nuanced in what way?}
\must{11}{Die Priorität einer getätigten Reservierung muss nuanciert angegeben werden können.}
\unsure{Wie genau soll das passieren? Entwerfen wir hierfür einen Algo?}
\must{12}{Die Anwendung muss Terminkonflikte reibungslos mithilfe der Prioritäten lösen können.}
\unsure{Wie genau soll das passieren?}
\must{13}{Die Anwendung muss in der Lage sein, Nutzer darüber zu informieren, wenn ihre Buchung durch eine Buchung mit höherer Priorität überschrieben wurde.}
\must{14}{Nutzer der Anwendung müssen in der Lage sein, eine Reservierung zu stornieren.}
\unsure{Hier sollten wir genauer spezifizieren, wie wir die passworverwaltung umsetzen wollen.}
\must{15}{Es muss einen Admin-Nutzer geben, welcher sich per Passwort sich authentifiziert. Alleine der Serveradmin darf dieses Passwort ändern können.}
\must{16}{Der Admin muss Termine löschen, Gastnutzer deaktivieren, Räume bearbeiten und Öffnungszeiten einstellen können.}
\must{17}{Die Anwendung muss den Raumstatus möglichst unkompliziert darstellen können. Es muss also sowohl die aktuelle Belegegung als auch die Priorität (z.B. durch eine farbigen Banner angezeigt), zu sehen sein.}

\section{Wunschkriterien}\label{sec:wunschkriterien}

\todo[inline]{Hier fallen uns noch sicherlich mehr Kriterien ein.}
\todo[inline]{Hier vielleicht mit KANN/KÖNNTE statt mit SOLL schreiben (Abstimmen: JA: 3, NEIN: 0)}

%\wish{1}{Es sollte die Möglichkeit geben mehr als einen Raum zur Buchung anzubieten. Dabei soll eine Arbeitsraum-Auswahl vor der Kalender Ansicht dem Nutzer die verschiedenen Möglichkeiten Präsentieren. Existiert nur ein Arbeitsraum wird diese Auswahl Übersprungen.}
\wish{1}{Es könnte die Möglichkeit geben mehr als einen Raum zur Buchung anzubieten. Dabei kann eine Arbeitsraum-Auswahl vor der Kalender Ansicht dem Nutzer die verschiedenen Möglichkeiten Präsentieren. Existiert nur ein Arbeitsraum wird diese Auswahl Übersprungen.}
\wish{2}{In der Kalenderansicht sollten Feiertage automatisch eingebunden werden.}
\wish{3}{Admins sollten die Möglichkeit haben geplante Wartungs- und Sperrzeiten einzurichten.}
\wish{4}{Termine sollen nach dem einrichten per iCal-Format exportiert werden können.}
\wish{5}{Nutzer sollen bei der Ereignis Ansicht die Möglichkeit haben ihre eigenen Termine zu editieren.}
\wish{6}{Die Kalenderansicht soll Anzeigen welche Termine bereits in der Vergangenheit liegen und wo der Übergang von Vergangenheit auf Zukunft liegt.}
\wish{7}{Tooltips sollen nutzern erklären wofür bestimmte Elemente der \gls{UI} sind.}
\wish{8}{Termine welche sich in Periodischem Takt wiederholen.}


\section{Abgrenzungskriterien}\label{sec:abgrenzungskriterien}
Abgrenzungskriterien: Leistungen die explizit nicht umgesetzt werden.\\
Hier ist zu verdeutlichen, welche Ziele mit dem Produkt bewusst nicht erreicht werden sollen oder können. 
Speziell sind hier Funktionen zur erwähnen, die sich der Kunde ursprünglich gewünscht (oder genannt) hat, die aber, nach Einigung, 
doch nicht umgesetzt werden sollen.
Auch Funktionen, die im Allgemeinen von ähnlichen Systemen zu erwarten wären,
hier aber explizit nicht umzusetzen sind (z.B.\ Login in einem Forum), sollten erwähnt werden.
Zu jedem System gehört normalerweise auch ein Benutzerhandbuch.
Wird ein Handbuch nicht benötigt, sollte dies hier festgehalten werden, sonst kann der Kunde später ein Handbuch verlangen.

\wont{1}{Die Verteilung von Buchungen zwischen ähnlichen Räumen ist nicht Teil des Projekts.}
\wont{2}{Das Skalieren der Anwendung auf große Raumzahlen ist nicht vorgesehen.}
\wont{3}{Die Buchung von Räumen für mehrere Tage ist nicht vorgesehen.}
\wont{4}{Die Verwaltung von Räumen, die mehrere Arbeitsplätze umfassen, ist nicht vorgesehen.}
\wont{5}{Die Entwicklung platfformspezifischer Anwendungen und der dafür notwendigen \gls{API}s ist nicht vorgesehen.}
\wont{6}{Die Reservierung ist nur für die nahe Zukunft gedacht, also keine Langzeitplanung.}