%!TEX root = ../Pflichtenheft.tex
\section{Allgemeine Hinweise}
\textbf{Template} Tragen Sie vor Beginn der Ausarbeitung Ihre Daten in die Datei 'common/teilnehmer.tex' ein. Löschen Sie die Template-Texte und -Bilder, sobald Sie Ihren Text geschrieben haben.

\textbf{Anglizismen} Verwenden Sie ruhig die englischen Wörter.

\textbf{Fett-/Kursivschreibweise} Entscheiden Sie sich zu Beginn innerhalb der Gruppe, wann Sie \textit{kursive} und wann Sie \textbf{fette} Schrift einsetzen möchten.

\textbf{Bilder} Legen Sie eigene Bilder stets im 'figures' Ordner ab. Nutzen Sie die im Template beispielhaft eingebundenen Bilder als Vorlage. Außerdem erlaubt Overleaf das Einfügen von Bildern im GUI (siehe obere Task-Leiste).


\chapter{Zielbestimmung}
\label{chap:target}
\improvement[inline]{Dieser Abschnitt sollte noch ausführlicher beschrieben werden.}
Dieser Abschnitt hat die Aufgabe, als eine Art Einleitung zu dienen. Es soll
ein kurzer Umriss über Ziel und Motivation des Gesamt- und ggf. der
Teilprojekte dargestellt werden. Beschrieben wird die Hauptaufgabe des Systems.
Wichtig ist, den Grund für die Systementwicklung (Probleme oder Geschäftsidee)
und damit ihre Ziele herauszuarbeiten.

\todo[inline]{The original todo note withouth chaflsjklfsdkljnged colours.\newline Here's another line.}
\section{Musskriterien}\label{sec:musskriterien}
In diesem Abschnitt halten wir unabdingbare Leistungen der Software fest\\
Wir halten also welche Funktionalitäten/Leistungen das Softwareprodukt in
jedem Fall erfüllen muss, damit es genutzt werden kann.
\todo[inline]{Hier fallen uns noch sicherliche mehr Kriterien ein.}
\must{1}{Die Anwendung soll als Web-Applikation realisiert werden.}
\must{2}{Die Anwendung soll schnell und einfach bedienbar sein.}
{\must{3}{Die Anwendung soll auf allen gängigen Browsern lauffähig sein.}}
\unsure{Kann man hier noch weiter spezifizieren?}{\must{4}{Es soll auf entsprechend Acessibility geachtet werden.}}
\must{5}{Es muss die Anonymität der Nutzer gewährleistet sein.}
\must{6}{User der Anwedung sollen in der Lage sein, einen Raum zu reservieren. Dabei sollte nicht nur eine einfache Kalendaransicht angeboten werden, sondern ein zusätzlichees Interface in dem über eine Raumbuchung verhandlen kann.}
\must{7}{Der Status einer getätigten Reservierung sollte nuanciert angegeben werden.}
\must{8}{Die Anwendung soll in der Lage sein, Benutzer*innen darüber zu informieren, wenn der Raum für einen Notfall reserviert wurde.}


\section{Sollkriterien}\label{sec:sollkriterien}
Sollkriterien: erstrebenswerte Leistungen\\
Dies sind Kriterien, die für die Lauffähigkeit des Produkts nicht zwingend
erforderlich sind, für die Erreichung der Projektziele aber erfüllt werden
sollten.

\should{1}{Das sollte auch realisiert werden.}
\should{2}{Und das ebenfalls.}

\section{Kannkriterien}\label{sec:kannkriterien}

Kannkriterien: Leistungen die enthalten sein können, denen der Auftraggeber jedoch neutral gegenüber steht.
Die Erfüllung dieser Kriterien ist nicht unbedingt notwendig, sie sollten nur angestrebt werden, 
falls noch ausreichend Kapazitäten vorhanden sind.

\could{1}{Wenn noch Zeit bleibt, könnte eine Statusanzeige über eine LED-Ampel über den Buchungsstatus realisiert werden können.}
\could{2}{Das würde ich mir wünschen, ist aber auch nicht so wichtig.}

\section{Abgrenzungskriterien}\label{sec:abgrenzungskriterien}
Abgrenzungskriterien: Leistungen die explizit nicht umgesetzt werden.\\
Hier ist zu verdeutlichen, welche Ziele mit dem Produkt bewusst nicht erreicht werden sollen oder können. 
Speziell sind hier Funktionen zur erwähnen, die sich der Kunde ursprünglich gewünscht (oder genannt) hat, die aber, nach Einigung, 
doch nicht umgesetzt werden sollen. Auch Funktionen, die im Allgemeinen von ähnlichen Systemen zu erwarten wären,
hier aber explizit nicht umzusetzen sind (z.B. Login in einem Forum), sollten erwähnt werden. 
Zu jedem System gehört normalerweise auch ein Benutzerhandbuch. Wird ein Handbuch nicht benötigt, 
sollte dies hier festgehalten werden, sonst kann der Kunde später ein Handbuch verlangen.

\wont{1}{Das will ich auf gar keinen Fall haben.}
\wont{2}{Dafür bezahle ich nicht.}
\wont{3}{Das wird schon durch ein anderes System abgedeckt.}