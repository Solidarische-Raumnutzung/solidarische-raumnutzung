%!TEX root = ../Pflichtenheft.tex

% Kapitel 4
%-------------------------------------------------------------------------------

\chapter{Produktfunktionen}
\label{chap:product_functions}

In Abhängigkeit von den gewählten Konzepten erfolgt hier eine Konkretisierung
und Detaillierung der Funktionen aus den Use-Case-Diagrammen und ggf. dem
Angebot.

Die Produktfunktionen müssen die Kriterien aus den Zielbestimmungen abdecken.
Dabei kann es je nach Kriterium eine oder mehrere Funktion geben.
Das nachfolgende Format sollte für einige Funktionsbeschreibungen übernommen werden:

Beispiel:\\
\begin{function}{10}{Lagerverwaltung}
\item[Anwendungsfall:] Automatisches Einlagern
\item[Anforderung:] \lfk{20} (Wenn kein Lastenheft vorhanden, dann Kriterium aus den Zielbestimmungen, z.B.: \ref{RM1})
\item[Ziel:] Ein Reifen erscheint am Systemeingang (Scanner), erhält einen Lagerplatz
zugewiesen und wird dort eingelagert.
\item[Vorbedingung:] Das Scannen des Barcode-Reifens muss erfolgreich sein, sonst kann
der Typ nicht ermittelt werden. Solche unbekannten Reifen werden direkt in den
Überlauf gefördert.
\item[Nachbedingung Erfolg:] Reifen ist physikalisch eingelagert und logisch in der
Datenbank verbucht.
\item[Nachbedingung Fehlschlag:] Der Reifen wurde infolge gestörter Fördermechanik
nicht eingelagert (liegt im Überlauf) oder produzierte aufgrund inkonsistenter
Datenbank einen "`Platz belegt"' - Fehler beim Anfahren eines irrtümlich
als frei angenommenen Platzes.
\item[Akteure:] ~Produktion
\item[Auslösendes Ereignis:] SPS meldet der Steuerung, dass am Eingangsscanner ein
Reifen mit Seriennummer X des Typs Y eingetroffen ist.
\item[Beschreibung:] ~
\begin{enumerate}
  \item Reifentypinformationen ermitteln (besonders Höhe des Reifens bei Wahl zwischen unterschiedlich hohen Lagerplätzen wichtig).
  \item Alle Module ermitteln, die\\
-  Platz auf den Einlagerstichen haben\\
-  momentan nicht im Störungszustand sind\\
-  freie Lagerplätze in der geforderten Höhe aufweisen.
  \item Lagerplatz nach Gleichverteilungsgrundsatz bestimmen.
  \item Reifen auf den Einlagerstich des gewählten Moduls befördern.
  \item Sobald er auf dem vordersten Platz des Einlagerstichs steht, dem Modul den Befehl zur Reifenaufnahme und Einlagerung auf den gewählten Platz schicken.
\end{enumerate}
\item[Erweiterung:] (optional)\\
	2a Zur Effizienzsteigerung auch Module ansteuern, die momentan keinen Platz auf
	den Einlagerstichen haben, aber wahrscheinlich so schnell einlagern, dass der
	Reifen nach der Fahrtzeit zum Modul auf den Stich eingelagert werden kann
	(Überwachung des "`Unterwegsbestandes"' an Reifen für ein bestimmtes
	Modul).\\
	3a Lagerplatz des Reifens möglichst nah zum Einlagerstich im RBG wählen	(kürzere RBG-Fahrtzeiten).
\item[Alternativen:] (optional)\\
	2a Wenn kein Lagerplatz gefunden wird, Reifen zum Überlauf schicken (der
	Einlagerförderer wird niemals angehalten!).
\end{function}

\begin{function}{20}{Login}
\item[Anwendungsfall:] ...
\end{function}