\chapter{Testfallbeschreibungen}
\label{chap:test}

Hier wird notiert, \textbf{was} und \textbf{wie} getestet wird. Dazu sollte sich an den Use-Cases und Funktionalitätsbeschreibungen orientiert werden.

\begin{testcase}{10}{Login}
  \item[Getestete Funktionalität] Authentifizierung registrierter Benutzer, <Anforderung> <Use-Case>, \ldots
  \item[Testform] Unit-Test. Der Unit-Test kann auch in Form einer Pipeline-Stufe mit \gls{CI} Workflows getestet werden \ldots
  \item[Vorbedingungen] Der zum Login willige Nutzer muss bereits registriert sein. Eine Registrierung umfasst u.a. die Angabe einer E-Mail-Adresse und das Wählen eines Passworts\ldots
  \item[Nachbedingungen] \ldots
  \item[Technik] Wie wird die Funktion getestet? Mock-Ups nötig? \ldots
\end{testcase}

\begin{testcase}{10}{Login}
  \item[Getestete Funktionalität] Authentifizierung registrierter Benutzer, <Anforderung> <Use-Case>, \ldots
  \item[Testform] Unit-Test. Der Unit-Test kann auch in Form einer Pipeline-Stufe mit \gls{CI} Workflows getestet werden \ldots
  \item[Vorbedingungen] Der zum Login willige Nutzer muss bereits registriert sein. Eine Registrierung umfasst u.a. die Angabe einer E-Mail-Adresse und das Wählen eines Passworts\ldots
  \item[Nachbedingungen] \ldots
  \item[Technik] Wie wird die Funktion getestet? Mock-Ups nötig? \ldots
\end{testcase}
