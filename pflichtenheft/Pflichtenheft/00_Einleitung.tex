%!TEX root = ../Pflichtenheft.tex
\chapter{Einleitung}
\todo[inline]{Das kann bestimmt ausschmücken... }
In modernen und inklusiven Arbeitsumgebungen spielen Rückzugsorte eine zentrale Rolle um produktives Arbeiten und und das Wohlbefinden aller gewährleisten zu könenn.
Das Institut für Mensch Maschine Interaktion (MMI) am KIT bietet eine Ruheraum, welcher für Meetings, kurzen Pausen oder als Rückzugsort nach einer Reizüberflutung genutzt werden kann.

Derzeit wird der Status des Raumes durch ein einfaches Türschild angezeigt, welches lediglich den Status 'verfügbar' oder 'besetzt' anzeigt.
Dieses System bietet jedoch keine Möglichkeit, die Art der Nutzung oder die Dringlichkeit des Bedarfs zu kommunizieren.

Im Rahmen dieses Projekts wird ein innovatives Buchungssystem entwickelt, das über ein klassisches 'frei'- oder 'gebucht'-System hinausgeht.
Ziel ist es, ein System zu schaffen, das es den Nutzer*innen ermöglicht, ihre Bedürfnisse differenziert anzugeben.
Dadurch soll eine transparente und flexible Abstimmung über die Raumnutzung ermöglicht werden, die sowohl individuellen Bedürfnissen als auch einer optimalen Raumverteilung gerecht wird.