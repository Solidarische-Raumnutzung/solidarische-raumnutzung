%!TEX root = ../main.tex

\chapter{Services}
\label{ch:services}

Der Service-Layer koordiniert die Logik der Applikation und stellt diese den Endpunkten, beispielsweise den Controllern, bereit.
Somit wird auch die Datenverarbeitung von der Verarbeitung der HTTP-Anfragen und Antworten abgekapselt.
Zur Bearbeitung dieser Aufgaben nutzen die Services den Repository-Layer, welcher die Interaktion mit der Datenbank abstrahiert.
Vor allem komplexere Aufgaben, wie das Buchen eines Termines, Auflösen eines Terminkonfliktes oder das
Versenden von E-Mails, werden von den Services behandelt.

Der \textit{Bookings Service} beinhaltet alle Logik für das Verwalten von Terminen: Löschen, Buchen, Koordination der Raumteilung, sowie das Vorbereiten der Termine, welche in der Kalenderansicht zu beobachten sind.

Der \textit{E-Mail Service} ermöglicht das Senden von E-Mails an die in den Konten hinterlegte E-Mail-Adresse.
Diese wurde je nach Kontotyp von den Nutzenden selber oder vom \gls{OIDC} Provider bereitgestellt.

Der \textit{Room Service} dient zur Auswahl des Raumes, um welchen sich z.B.\ eine Buchung dreht.
Diese Klasse ermöglicht insbesondere eine einfache Erweiterung der Applikation auf mehrere Räume.

Der \textit{System Configuration Service} dient zur Einstellung der Applikation bereitgestellten Funktionen, so wird hier eingestellt, ob Gastkonten aktiviert sind.
Wird von einem Admin die Gastkonten funktion deaktiviert, so werden von diesem Service auch die Termine aller betroffenen Konten gelöscht.

Der \textit{User Service} hilft mit dem Umgang der Konten: Das Deaktivieren eines Kontos, überprüfen, ob ein Konto Admin oder Gaststatus besitzt und erstellen sowie Löschen von Konten.


\InputIfFileExists{javadoc/edu.kit.hci.soli.service}{}{}
