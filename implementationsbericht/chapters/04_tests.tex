%!TEX root = ../main.tex

\chapter{Tests und Coverage}
\label{ch:tests}

Für kontinuierliches Testen und zur Sicherstellung der Qualität des Codes wurden sowohl Unit- als auch Integrationstests geschrieben.
Diese werden nach jedem Push auf das Repository durch GitHub CI automatisiert ausgeführt.

Umgesetzt sind die Tests mit JUnit als Testframework, Mockito für Mocking und werden von Gradle gestartet.
Damit sichergestellt werden kann, dass die Tests in einem Realitätsnahen Umfeld laufen,
wird eine normale Postgres-Datenbank verwendet, welche in einem von Gradle gestarteten Docker-Container läuft.

Um die Abdeckung unseres Quellcodes durch die Tests sicherzustellen, setzen wir JaCoCo ein.
JaCoCo wird ebenfalls von Gradle gestartet und generiert nach jedem Testdurchlauf einen Report, der in GitHub CI angezeigt wird.
Bei einer zu niedrigen Testabdeckung wird zudem ein fehlgeschlagener Check in GitHub CI generiert.

Ausgenommen von den Tests ist lediglich der Inhalt des Pakets \textit{config},
da dieser lediglich der Konfiguration von Spring dient und nicht gut durch Unit- oder Integrationstests geprüft werden kann, sowie \textit{view} da dies schwer automatisiert visuell zu überprüfen.

\newpage

\section{Coverage}\label{sec:coverage}

Zum Ende der Implementationsphase haben wir in mehr als 150 Tests eine Coverage von 67\% erreicht.
Diese gliedert sich wie folgt in die verschiedenen Pakete auf:

\begin{table}[h]
    \centering
    \renewcommand{\arraystretch}{1.3}
    \begin{tabular}{l|c}
        \textbf{Paket} & \textbf{Line Coverage} \\
        \hline
        \hline
        \textit{Controller}  & 69\% \\
        \textit{Domain}      & 87\% \\
        \textit{DTO}         & 79\% \\
        \textit{Filter}      & 100\% \\
        \textit{Repository}  & 100\% \\
        \textit{Service}     & 85\% \\
        \hline
        \textit{Gesamt}      & 67\% \\
    \end{tabular}
    \caption{Coverage der verschiedenen Pakete}
    \label{tab:progress}
\end{table}
