%!TEX root = ../main.tex

\chapter{Kriterien}
\label{chap:kriterien}

In diesem Kapitel wird auf die Musskriterien eingegangen, die im Pflichtenheft spezifiziert wurden.
Für jedes Kriterium wird angegeben, ob es im Rahmen der Implementierung umgesetzt wurde,
ob es in der Umsetzung geändert wurde und falls ja, wie es geändert wurde,
oder ob es nicht umgesetzt wurde und warum.

\section{Musskriterien}\label{sec:musskriterien}

Die Funktionalität aller Musskriterien wurde in der Implementierung umgesetzt.
Es gibt allerdings kleine Anpassungen, die im Folgenden beschrieben werden.

\must{1}{Die Anwendung wurde als Web-Applikation realisiert.}
\must{2}{Nutzende der Anwendung können sich mit ihrem KIT-Konto per \gls{OIDC} oder einem lokalen Gastkonto anmelden.}
\must{3}{Nutzende können sich abmelden.}
\must{4}{Die im Pflichtenheft beschriebenen Ansichten sind vorhanden.}
\must{5}{Die Ansicht \textit{Kalender} gibt mit FullCalendar-Events einen Überblick über die reservierten Zeiten.}
\must{6}{Die Ansicht \textit{Kalender} stellt die Öffnungszeiten des Raumes farblich dar.}
\must{7}{Die Ansicht \textit{Kalender} hebt die Termine des angemeldeten Nutzenden mit einem Badge hervor.}
\must{8}{Die Ansicht \textit{Termin} zeigt sowohl Start- und Endzeitpunkt als auch Beschreibung des Termins.}
\must{9}{Die Ansicht \textit{Termin-Erstellen} ermöglicht es, einen Raum für eine bestimmte Zeitperiode zu reservieren.}
\must{10}{Bei der Reservierung eines Raumes kann optional eine Beschreibung hinterlegt werden. Ein Paragraph informiert über die Sichtbarkeit dieser Beschreibung.}
\must{11}{Die Ansicht \textit{Terminübersicht} zeigt alle Termine des angemeldeten Nutzenden in einer Tabelle an.}
\must{12}{Die Priorität eines Termins ist in drei Stufen gegliedert.}
\must{13}{Beim Erstellen eines Termins wählen die Nutzenden aus, ob sie ihren Raum mit anderen Personen teilen möchten. Dabei stehen die Optionen \textit{Ja}, \textit{Nein} und \textit{Auf Anfrage} zur Auswahl.}
\must{14}{Die Anwendung löst Terminkonflikte mithilfe der Prioritäten. Termine mit höherer Priorität überschreiben andere. Nutzende werden dabei über die Überschreibung informiert.}
\must{15}{Nutzende werden per E-Mail informiert, wenn ihr Termin durch einen Termin mit höherer Priorität überschrieben wurde.}
\must{16}{Nutzende können eine Reservierung stornieren. Dies ist über die Terminansicht mit dem Button \textit{Löschen} möglich.}
\must{17}{Es gibt ein Adminkonto, welches per Passwort authentifiziert wird. Es wird per Docker-Compose festgelegt und kann also nur durch den Server-Admin geändert werden.}
\must{18}{Die Ansicht \textit{Kontoliste} ermöglicht es Admins, einzelne Konten sowie die Anmeldung per Gastkonto zu deaktivieren.}
\must{19}{Admins können in der Ansicht \textit{Termin} Termine löschen.}
\must{20}{Ein farbcodiertes Banner zeigt den aktuellen Status des Raumes an.}
\must{21}{Admins können Öffnungszeiten einstellen, für diese Möglichkeit gibt es allerdings eine neue Ansicht.}

\section{Wunschkriterien}\label{sec:wunschkriterien}

Einige Wunschkriterien wurden ebenfalls umgesetzt.
Im Folgenden wird beschrieben, welche Wunschkriterien umgesetzt wurden und welche nicht.

\wish{1}{Es gibt die Möglichkeit, mehr als einen Raum zur Buchung anzubieten. Die zur Verfügung stehenden Räume können über die Ansicht \textit{Räume} konfiguriert werden. Dem Nutzer wird falls nötig eine Raumauswahl vor der Ansicht \textit{Kalender} präsentiert. Existiert nur ein Raum, wird diese Auswahl übersprungen.}
\wish{2}{In der Ansicht \textit{Kalender} werden Feiertage automatisch eingebunden.}
\wish{3}{Geplante Warn- und Sperrzeiten wurden nicht umgesetzt, da sie nicht als notwendig erachtet wurden.}
\wish{4}{Termine können nach der Buchung im \gls{iCal}-Format zum Export angeboten werden.}
\wish{5}{Nutzende können in der Ansicht \textit{Termin} die Beschreibung ihrer Termine bearbeiten. Um die Oberfläche einfach zu halten, wurde die Bearbeitung anderer Felder nicht umgesetzt.}
\wish{6}{Die Ansicht \textit{Kalender} visualisiert, welche Termine bereits in der Vergangenheit liegen und wo der Übergang von der Vergangenheit zur Zukunft liegt. Dafür wird die aktuelle Zeit mit einem roten Strich markiert.}
\wish{7}{Tooltips erklären Nutzenden, wofür die wichtigsten Elemente der \gls{UI} verwendet werden. Im Verlauf der QA-Phase sollen diese basierend auf Nutzerfeedback weiter verbessert werden.}
\wish{9}{Ein physischer Panik-Button wurde nicht umgesetzt, da er nicht als notwendig erachtet wurde.}
\wish{10}{Die Anwendung informiert Nutzende nicht, wenn ein gewünschter Termin frei wird, da keine überzeugende Kombination aus Use-Case und guter Nutzererfahrung gefunden wurde.}
\wish{11}{Ein Quick-Checkin-Button wurde umgesetzt. Der Aktuelle Zeitslot ist bei Nutzung des Buttons bereits vorausgefüllt.}
\wish{12}{Ein Quick-Checkout-Button wurde umgesetzt. Der Nutzer kann den aktuellen Termin vorzeitig beenden.}
\wish{13}{Admins können eine globale Statistik-Ansicht nutzen um die Buchungen pro Wochentag, Tag des Monats und Monat anzuzeigen.}
